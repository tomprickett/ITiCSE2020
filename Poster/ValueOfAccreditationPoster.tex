% The first command in your LaTeX source must be the \documentclass command.

\documentclass[sigconf]{acmart}
 % Do not change for ITiCSE'20

\settopmatter{printacmref=true}
  % mandatory for ITiCSE'20
\settopmatter{authorsperrow=4}

\fancyhead{}
  % do not delete this code.

\usepackage{balance}
\usepackage{paralist}
  % for creating a balanced last page (usually last page with references)

% defining the \BibTeX command - from Oren Patashnik's original BibTeX documentation.
\def\BibTeX{{\rm B\kern-.05em{\sc i\kern-.025em b}\kern-.08emT\kern-.1667em\lower.7ex\hbox{E}\kern-.125emX}}
    
% Rights management information. 
% This information is sent to you when you complete the rights form.
% These commands have SAMPLE values in them; it is your responsibility as an author to replace
% the commands and values with those provided to you when you complete the rights form.
%
% These commands are for a PROCEEDINGS abstract or paper.

% Submission ID. 
% Use this when submitting an article to a sponsored event. You'll receive a unique submission ID from the organizers
% of the event, and this ID should be used as the parameter to this command.
\acmSubmissionID{pp059}


% end of the preamble, start of the body of the document source.
\setcopyright{rightsretained}
\begin{document}
\copyrightyear{2020}
\acmYear{2020}
\acmConference[ITiCSE '20]{Proceedings of the 2020 ACM Conference on Innovation and Technology in Computer Science Education}{June 15--19, 2020}{Trondheim, Norway}
\acmBooktitle{Proceedings of the 2020 ACM Conference on Innovation and Technology in Computer Science Education (ITiCSE '20), June 15--19, 2020, Trondheim, Norway}
\acmDOI{10.1145/3341525.3393980}
\acmISBN{978-1-4503-6874-2/20/06}


\fancyhead{}
  % do not delete this code.


% The "title" command has an optional parameter, allowing the author to define a "short title" to be used in page headers.
\title{Assessing the Value of Professional Body Accreditation of Computer Science Degree Programmes: A UK Case Study}


% The "author" command and its associated commands are used to define the authors and their affiliations.
% Of note is the shared affiliation of the first two authors, and the "authornote" and "authornotemark" commands
% used to denote shared contribution to the research.
\author{Tom Crick}
\orcid{0000-0001-5196-9389}
\affiliation{%
	\institution{Swansea University}
	\city{Swansea}
	\country{UK}
}
\email{thomas.crick@swansea.ac.uk}

\author{Tom Prickett}
\orcid{0000-0002-9671-2250}
\affiliation{%
	\institution{ Northumbria University}
	\city{Newcastle upon Tyne}
	\country{UK}
}
\email{tom.prickett@northumbria.ac.uk}

\author{James H. Davenport}
\orcid{0000-0002-3982-7545}
\affiliation{%
	\institution{ University of Bath}
	\city{Bath}
	\country{UK}
}
\email{masjhd@bath.ac.uk}


% \begin{comment}


% \author{Paul Hanna}
% \affiliation{%
% \institution{Ulster University}
% \city{Belfast}
% \country{UK}
% }
% \email{jrp.hanna@ulster.ac.uk}

% \end{comment}

\author{Alastair Irons}
\affiliation{%
	\institution{ Sunderland University}
	\city{Sunderland}
	\country{UK}
}
\email{alastair.irons@sunderland.ac.uk}


%
% By default, the full list of authors will be used in the page headers. Often, this list is too long, and will overlap
% other information printed in the page headers. This command allows the author to define a more concise list
% of authors' names for this purpose.
\renewcommand{\shortauthors}{Crick et al.}

%
% The abstract is a short summary of the work to be presented in the article.
\begin{abstract}
This poster presents a model for the value provided by professional
body accreditation of computer science degree programmes in the United
Kingdom (UK). We introduce how one large UK professional computing
body -- BCS, The Chartered Institute for IT (BCS)-- addresses degree
accreditation, as well as recent changes to content and process. Whilst
comparable accreditation regimes exist in a number of other
jurisdictions, we provide the opportunity for exploring future
extensions to, and the portability of, the UK model.

\end{abstract}

%
% The code below is generated by the tool at http://dl.acm.org/ccs.cfm.
% Please copy and paste the code instead of the example below.
%
\begin{CCSXML}
	<ccs2012>
	<concept>
	<concept_id>10003456.10003457.10003527.10003529</concept_id>
	<concept_desc>Social and professional topics~Accreditation</concept_desc>
	<concept_significance>500</concept_significance>
	</concept>
	<concept>
	<concept_id>10003456.10003457.10003527.10003531</concept_id>
	<concept_desc>Social and professional topics~Computing education programs</concept_desc>
	<concept_significance>300</concept_significance>
	</concept>
	<concept>
	<concept_id>10003456.10003457.10003580.10003568</concept_id>
	<concept_desc>Social and professional topics~Employment issues</concept_desc>
	<concept_significance>100</concept_significance>
	</concept>
	</ccs2012>
\end{CCSXML}

\ccsdesc[500]{Social and professional topics~Accreditation}
\ccsdesc[300]{Social and professional topics~Computing education programs}
\ccsdesc[100]{Social and professional topics~Employment issues}


%%
%% Keywords. The author(s) should pick words that accurately describe
%% the work being presented. Separate the keywords with commas.
\keywords{Accreditation; Professional Body; Curricula Design}

%
% This command processes the author and affiliation and title information and builds
% the first part of the formatted document.
\maketitle

%\section*{The Model}

\textbf{ Background Research.} The value of professional body degree accreditation regimes as a
kite-marking exercise (or to support a globally-portable and
recognised workforce) remains high~\cite{Knight_2015}. Equally, the
respective national (or otherwise) regimes are criticised as: being unnecessarily bureaucratic and  
constraining innovation ~\cite{Harvey2004};
generating revenues streams in their own right rather than for the
benefit of a discipline or wider society~\cite{Knight_2015}; or
colonial and paternalistic in nature~\cite{Mutereko2017}. The value provided by one such accreditation (BCS) regime was assessed by the views of UK higher education institutions (HEIs) by: canvassing as part of 18 formal accreditation visits (from
September 2018-September 2019); workshops run between November 2018
and November 2019 as part of the operation of BCS Academic
Accreditation Committee (AAC); and survey-based feedback gained from
BCS Academic Assessors, attendees of the 2020 ACM Computing Education
Practice Conference~\cite{CrickEtAl2020Cep} and the
readership of ITNow~\cite{CrickEtAl2020ITNow}, the `the voice of the
BCS' which publishes articles on all aspects of computing and IT.  

\textbf{The Proposed Model}   The following are the value of accreditation: Raising output standards, essentially performing a
		kite-marking function; employ internationally-recognised standards and
		memoranda (e.g. Seoul Accord, Washington Accord, EQANIE) to
		promote the global parity of computer science education and the
		mobility of graduates;	ensuring curricula relevance e.g. coverage of
		cybersecurity~\cite{Cricketal2019}, team working and professional
		environment;
	identifying and disseminating practice highlights
		either directly \cite{practice_highlights_2020} or by other means ~\cite{CrickEtAl2020Cep}
	; promoting Industry relevant curricula; 
	 accrediting work experience in degree programmes; and responding to criticism (reduce administrative burden, promote innovation, etc).
	 Generic criticisms of professional body accreditation regimes not withstanding, the feedback to date is broadly supportive of the proposed model.


%\section*{Future Work}

\textbf{Future Work.} The next step is to compare and contrasts the model to that used in other jurisdictions and assess portability. The value of accreditation is linked to the value of membership and registration with a professional body, the enhancement of this value to students, graduates, early career professionals and academics is currently being explored by the BCS.  

%%
%% The acknowledgments section is defined using the "acks" environment
%% (and NOT an unnumbered section). This ensures the proper
%% identification of the section in the article metadata, and the
%% consistent spelling of the heading.
% \newpage
% \balance
\begin{acks}
Thanks to the supporting BCS Volunteers and Accreditation Team.  
The authors' institutions are members of the Institute of Coding, an initiative funded by the Office for Students (England) and the Higher Education Funding Council for Wales.	
\end{acks}
%%
%% The next two lines define the bibliography style to be used, and
%% the bibliography file.
\bibliographystyle{ACM-Reference-Format}
%\bibliography{sample-base}
\bibliography{poster}
%%
%% If your work has an appendix, this is the place to put it.
\appendix

\end{document}
