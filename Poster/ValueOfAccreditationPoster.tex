% The first command in your LaTeX source must be the \documentclass command.

\documentclass[sigconf]{acmart}
 % Do not change for ITiCSE'20
\settopmatter{authorsperrow=4}
\settopmatter{printacmref=true}
  % mandatory for ITiCSE'20

\fancyhead{}
  % do not delete this code.

\usepackage{balance}
  % for creating a balanced last page (usually last page with references)

% defining the \BibTeX command - from Oren Patashnik's original BibTeX documentation.
\def\BibTeX{{\rm B\kern-.05em{\sc i\kern-.025em b}\kern-.08emT\kern-.1667em\lower.7ex\hbox{E}\kern-.125emX}}
    
% Rights management information. 
% This information is sent to you when you complete the rights form.
% These commands have SAMPLE values in them; it is your responsibility as an author to replace
% the commands and values with those provided to you when you complete the rights form.
%
% These commands are for a PROCEEDINGS abstract or paper.

\copyrightyear{2020}
\acmYear{2020}
\setcopyright{acmcopyright}
   % adjust this to the correct options per the rightsreview. Provided in ACM rightsreview confirmation email.
\acmConference[ITiCSE '20] {2020 ACM Conference on Innovation and Technology in Computer Science Education}{June 15--19, 2020}{Trondheim, Norway}
\acmBooktitle{2020 ACM Conference on Innovation and Technology in Computer Science Education (ITiCSE'20), June 15--19, 2020, Trondheim, Norway}
\acmPrice{15.00}
\acmDOI{10.1145/XXXXXX.XXXXXX}
  % edit the X's to your assigned DOI. Providing in ACM rightsreview confirmation email.
\acmISBN{978-1-4503-6874-2/20/06} 


% Submission ID. 
% Use this when submitting an article to a sponsored event. You'll receive a unique submission ID from the organizers
% of the event, and this ID should be used as the parameter to this command.
%\acmSubmissionID{123-A56-BU3}


% end of the preamble, start of the body of the document source.

\begin{document}

\fancyhead{}
  % do not delete this code.


% The "title" command has an optional parameter, allowing the author to define a "short title" to be used in page headers.
\title{Value of Accreditation of Computer Science Degrees by Professional Bodies:  Is the model generated by a UK Case Study Portable?}


% The "author" command and its associated commands are used to define the authors and their affiliations.
% Of note is the shared affiliation of the first two authors, and the "authornote" and "authornotemark" commands
% used to denote shared contribution to the research.

\author{Tom Crick}
\orcid{0000-0001-5196-9389}
\affiliation{%
	\institution{Swansea University}
	\city{Swansea}
	\country{UK}
}
\email{thomas.crick@swansea.ac.uk}

\author{Tom Prickett}
\affiliation{%
	\institution{ Northumbria University}
	\city{Newcastle upon Tyne}
	\country{UK}
}
\email{tom.prickett@northumbria.ac.uk}

\author{James H. Davenport}
\orcid{0000-0002-3982-7545}
\affiliation{%
	\institution{ University of Bath}
	\city{Bath}
	\country{UK}
}
%%\email{j.h.davenport@bath.ac.uk}


\begin{comment}


\author{Paul Hanna}
\affiliation{%
\institution{Ulster University}
\city{Belfast}
\country{UK}
}
\email{jrp.hanna@ulster.ac.uk}

\end{comment}

\author{Alastair Irons}
\affiliation{%
	\institution{ Sunderland University}
	\city{Sunderland}
	\country{UK}
}
%%\email{alastair.irons@sunderland.ac.uk}


%
% By default, the full list of authors will be used in the page headers. Often, this list is too long, and will overlap
% other information printed in the page headers. This command allows the author to define a more concise list
% of authors' names for this purpose.
\renewcommand{\shortauthors}{Trovato and Tobin, et al.}

%
% The abstract is a short summary of the work to be presented in the article.
\begin{abstract}
	This poster presents a model for the value provided by professional body accreditation of Computer Science Degrees, in one jurisdiction namely the United Kingdom (UK). The proposed model is that of The BCS, The Chartered Institute for IT (BCS). Parable accreditation regimes exist in a number of other jurisdictions, providing the opportunity for exploring the portability of and extensions to the proposed model.
\end{abstract}

%
% The code below is generated by the tool at http://dl.acm.org/ccs.cfm.
% Please copy and paste the code instead of the example below.
%
 \begin{CCSXML}
	<ccs2012>
	<concept>
	<concept_id>10003456.10003457.10003527.10003529</concept_id>
	<concept_desc>Social and professional topics~Accreditation</concept_desc>
	<concept_significance>500</concept_significance>
	</concept>
	<concept>
	<concept_id>10003456.10003457.10003527.10003531</concept_id>
	<concept_desc>Social and professional topics~Computing education programs</concept_desc>
	<concept_significance>300</concept_significance>
	</concept>
	<concept>
	<concept_id>10003456.10003457.10003580.10003568</concept_id>
	<concept_desc>Social and professional topics~Employment issues</concept_desc>
	<concept_significance>100</concept_significance>
	</concept>
	</ccs2012>
\end{CCSXML}

\ccsdesc[500]{Social and professional topics~Accreditation}
\ccsdesc[300]{Social and professional topics~Computing education programs}
\ccsdesc[100]{Social and professional topics~Employment issues}

%
% Keywords. The author(s) should pick words that accurately describe the work being
% presented. Separate the keywords with commas.
\keywords{Accreditation, Professional Body, Curricula Design}

%
% This command processes the author and affiliation and title information and builds
% the first part of the formatted document.
\maketitle

\section{The Model}
The value of professional body higher education accreditation regimes as a kite-marking exercise or to support a globally mobile workforce \cite{Knight_2015} remains high. Equally the regimes are criticised for being perceived to be: unnecessarily bureaucratic and constraining innovation \cite{Harvey2004}; revenues streams in their own right rather than for the benefit of a discipline or wider society \cite{Knight_2015}; or colonial \cite{Mutereko2017}.  
The model presented in this poster was generated by insights gained from the views of Higher Education Institutes (HEI's) canvased as part of 18 BCS  accreditation visits (from September 2018-September 2019), workshops ran in November 2018 and November 2019 as part of the operation of BCS Academic Accreditation Committee (AAC) and survey based feedback gained from BCS Academic Assessors,  attendees of the ACM Computing Education Practice Conference \cite{CrickEtAl2020Cep} and the readership of ITNow \cite{CrickEtAl2020ITNow}.
The proposed model represents a commitment to continuously review and enhance practices in response to criticism (e.g. remove bureaucracy, further support graduate employment, etc.) and to enhance the value provided. The following are aspects of the current value to accredited HEI's.
\begin{itemize}
	\item {Raising output standards, essentially performing a kite-marking function.}
	\item {Employ internationally-recognised standards and memoranda (e.g. Seoul Accord, Washington, Washington Accord European Quality Assurance Network for Informatics Education (EQANIE)) to promote the global parity of Compute Science education and hence the mobility of graduates.} 
	\item {Ensuring curricula relevance e.g. coverage of cybersecurity \cite{Cricketal2019}, team working and professional environment}
	\item {Identifying and disseminating practice highlights either directly \cite{practice_highlights_2020} or via other means such as conferences (e.g. ACM Computer Education Practice Conference \cite{CrickEtAl2020Cep}} )
	\item {Industry relevance by mandating the inclusion of industrialist upon accreditation panels} 
	\item {Accrediting work experience in degree programmes.}
\end{itemize}
\section{Future work}
The feedback to date indicates that generic criticisms of professional body accreditation regimes not withstanding, the developed model is broadly accepted in the UK jurisdiction. The next step is to explore the portability and applicability of this model to other jurisdictions.  

\begin{comment}


\subsection{Acknowledgements}
Many have contributed to this work: BCS Assessors; visited BCS Education Affiliate HEI's; respondants to surveys; BCS AAC and BCS Accreditation Team.

\end{comment}


%
% The next two lines define the bibliography style to be used, and the bibliography file.
\bibliographystyle{ACM-Reference-Format}
\bibliography{sample-base}

% 
% If your work has an appendix, this is the place to put it.

\end{document}
