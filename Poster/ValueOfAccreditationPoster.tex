%%
%% This is file `sample-authordraft.tex',
%% generated with the docstrip utility.
%%
%% The original source files were:
%%
%% samples.dtx  (with options: `authordraft')
%% 
%% IMPORTANT NOTICE:
%% 
%% For the copyright see the source file.
%% 
%% Any modified versions of this file must be renamed
%% with new filenames distinct from sample-authordraft.tex.
%% 
%% For distribution of the original source see the terms
%% for copying and modification in the file samples.dtx.
%% 
%% This generated file may be distributed as long as the
%% original source files, as listed above, are part of the
%% same distribution. (The sources need not necessarily be
%% in the same archive or directory.)
%%
%% The first command in your LaTeX source must be the \documentclass command.
%%\documentclass[sigconf,authordraft]{acmart}

\documentclass[sigconf]{acmart}
\settopmatter{authorsperrow=4}

%%
%% \BibTeX command to typeset BibTeX logo in the docs
\AtBeginDocument{%
  \providecommand\BibTeX{{%
    \normalfont B\kern-0.5em{\scshape i\kern-0.25em b}\kern-0.8em\TeX}}}

%% Rights management information.  This information is sent to you
%% when you complete the rights form.  These commands have SAMPLE
%% values in them; it is your responsibility as an author to replace
%% the commands and values with those provided to you when you
%% complete the rights form.
\setcopyright{acmcopyright}
\copyrightyear{2020}
\acmYear{2020}
\acmDOI{10.1145/1122445.1122456}

%% These commands are for a PROCEEDINGS abstract or paper.
\acmConference[Trondheim 2020]{Trondheim 2020: ACM Innovation and Technology in Computer Science Education (ITiCSE)}{June 17--18, 2020}{Trondheim, Norway}
\acmBooktitle{Trondheim 2020: ACM Innovation and Technology in Computer Science Education,
  June 17--18, 2020, Trondheim, Norway}
\acmPrice{15.00}
\acmISBN{978-1-4503-9999-9/18/06}




%%
%% Submission ID.
%% Use this when submitting an article to a sponsored event. You'll
%% receive a unique submission ID from the organizers
%% of the event, and this ID should be used as the parameter to this command.
%%\acmSubmissionID{123-A56-BU3}

%%
%% The majority of ACM publications use numbered citations and
%% references.  The command \citestyle{authoryear} switches to the
%% "author year" style.
%%
%% If you are preparing content for an event
%% sponsored by ACM SIGGRAPH, you must use the "author year" style of
%% citations and references.
%% Uncommenting
%% the next command will enable that style.
%%\citestyle{acmauthoryear}

%%
%% end of the preamble, start of the body of the document source.
\begin{document}

%%
%% The "title" command has an optional parameter,
%% allowing the author to define a "short title" to be used in page headers.
\title{Accreditation of Computer Science Degrees by Professional Bodies:  Is the model of Value generated by a UK Case Study Portable?}

\author{Tom Crick}
\orcid{0000-0001-5196-9389}
\affiliation{%
\institution{Swansea University}
\city{Swansea}
\country{UK}
 }
\email{thomas.crick@swansea.ac.uk}

\author{Tom Prickett}
\affiliation{%
 \institution{ Northumbria University}
 \city{Newcastle upon Tyne}
 \country{UK}
 }
\email{tom.prickett@northumbria.ac.uk}



\author{James H. Davenport}
\orcid{0000-0002-3982-7545}
\affiliation{%
\institution{ University of Bath}
\city{Bath}
\country{UK}
}
%%\email{j.h.davenport@bath.ac.uk}

\begin{comment}

\author{Paul Hanna}
\affiliation{%
\institution{Ulster University}
\city{Belfast}
\country{UK}
}
\email{jrp.hanna@ulster.ac.uk}
\end{comment}

\author{Alastair Irons}
\affiliation{%
\institution{ Sunderland University}
 \city{Sunderland}
\country{UK}
}
%%\email{alastair.irons@sunderland.ac.uk}




%%
%% By default, the full list of authors will be used in the page
%% headers. Often, this list is too long, and will overlap
%% other information printed in the page headers. This command allows
%% the author to define a more concise list
%% of authors' names for this purpose.
\renewcommand{\shortauthors}{Crick, Davenport, Hanna, Irons, and Prickett}

%%
%% The abstract is a short summary of the work to be presented in the
%% article.
\begin{abstract}
  In this poster a model for the value provided by professional body accreditation is presented. This is from one jurisdiction namely the the United Kingdom (UK). The proposed model, is that of one professional body, The BCS, The Chartered Institute for IT. 
\end{abstract}

 \begin{CCSXML}
<ccs2012>
<concept>
<concept_id>10003456.10003457.10003527.10003529</concept_id>
<concept_desc>Social and professional topics~Accreditation</concept_desc>
<concept_significance>500</concept_significance>
</concept>
<concept>
<concept_id>10003456.10003457.10003527.10003531</concept_id>
<concept_desc>Social and professional topics~Computing education programs</concept_desc>
<concept_significance>300</concept_significance>
</concept>
<concept>
<concept_id>10003456.10003457.10003580.10003568</concept_id>
<concept_desc>Social and professional topics~Employment issues</concept_desc>
<concept_significance>100</concept_significance>
</concept>
</ccs2012>
\end{CCSXML}

\ccsdesc[500]{Social and professional topics~Accreditation}
\ccsdesc[300]{Social and professional topics~Computing education programs}
\ccsdesc[100]{Social and professional topics~Employment issues}

%%
%% Keywords. The author(s) should pick words that accurately describe
%% the work being presented. Separate the keywords with commas.
\keywords{Accreditation, Professional Body, Curricula Design}



%% A "teaser" image appears between the author and affiliation
%% information and the body of the document, and typically spans the
%% page.

%%
%% This command processes the author and affiliation and title
%% information and builds the first part of the formatted document.
\maketitle

\section{Introduction}
The value of professional body higher education accreditation regimes as a kite-marking exercise or to support a globally mobile workforce \cite{Knight_2015} remains high. Equally the regimes are criticised for being unnecessarily bureaucratic and constraining innovation \cite{Harvey2004}, or perceived as revenues streams in their own right rather than for the benefit of a discipline or wider society \cite{Knight_2015} or even colonial \cite{Mutereko2017}.  This poster presents the proposed model with a view to exploring its potential portability to other jurisdictions.

The model was generated by insights gained from the views of Higher Education Institutes (HEI's) canvased as parted of accreditation visits (from September 2018-September 2019), workshops ran in November 2018 and November 2019 as part of the operation of BCS Academic Accreditation Committee (AAC) and survey based feedback gained from BCS Academic Assessors,  attendees of the ACM Computing Education Practice Conference \cite{CrickEtAl2020Cep} and the readership of ITNow \cite{CrickEtAl2020ITNow}

The following are aspects of the value to accredited HEIs.

\begin{itemize}

    
    \item {Raising output standards, essentially performing a kite-marking function.}

\item {Promoting internationally-recognised standards, global parity of Compute Science education and hence the mobility of graduates.} 

\item {Ensuring curricula relevance for example coverage of cybersecurity \cite{Cricketal2019}, team working and professional environment}

\item {Identifying and disseminating practice highlights either directly \cite{practice_highlights_2020} or via other means suck as conferences (for example ACM Computer Education Practice Conference \cite{CrickEtAl2020Cep}} )

\item {Industry relevance by mandating the inclusion of industrialist upon accreditation panels} 

\item {Accrediting work experience in degree programmes.}
\end{itemize}

It is also incumbent upon professional bodies to continuously review and enhance their policies and procedures. Part of the process must be to where possible reduce the administrative burden upon HEIs.

\section{Future work}
The feedback to date indicate that generic criticisms of professional body accreditation regimes not withstanding, the developed model is broadly accepted in the UK jurisdiction. The next step is to explore the portability and applicability of this models to other jurisdictions.  

\begin{comment}


\subsection{Acknowledgements}
Many have contributed to this work: BCS Assessors; visited BCS Education Affiliate HEI's; respondants to surveys; BCS AAC and BCS Accreditation Team.

\end{comment}
%%
%% The next two lines define the bibliography style to be used, and
%% the bibliography file.
\bibliographystyle{ACM-Reference-Format}
\bibliography{sample-base}

%%
%% If your work has an appendix, this is the place to put it.
\appendix


\end{document}
\endinput
%%
%% End of file `sample-authordraft.tex'.
